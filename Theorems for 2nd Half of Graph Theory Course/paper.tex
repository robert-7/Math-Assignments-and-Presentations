\documentclass[12pt]{amsart}

\usepackage{enumerate, color, amssymb, amsmath, amsfonts, amsthm, graphics, mathrsfs}
%\usepackage[hmargin=1 in, vmargin = 0.5 in]{geometry}

%%%%%%% Pagestyle stuff %%%%%%%%%%%%%%%%%%%
 \usepackage{fancyhdr}                   %%
   \pagestyle{fancy}                     %%
   \fancyhf{} %delete the current section for header and footer
 \usepackage[paperheight=11in,%          %%
             paperwidth=8.5in,%          %%
             outer=1.2in,%               %%
             inner=1.2in,%               %%
             bottom=.7in,%               %%
             top=.7in,%                  %%
             includeheadfoot]{geometry}  %%
   \addtolength{\headwidth}{.75in}       %%
   \fancyhead[RO,LE]{\thepage}           %%
   \fancyhead[RE,LO]{\sectionname}       %%
   \setlength{\headheight}{15.8pt}       %%
   \raggedbottom                         %%
%%%%%%% End Pagestyle stuff %%%%%%%%%%%%%%%

% When you first define a new word, use this macro to make it stand out
% EG We say that an abelian group $I$ is \newword{injective} if, for any
% injection $G \to H$, and any map $G \to I$, there is a map $H \to I$ making the
% obvious diagram commute.
\newcommand{\newword}[1]{\textbf{\emph{#1}}}

%Arrows
\newcommand{\into}{\hookrightarrow}
\newcommand{\onto}{\twoheadrightarrow}

%Things LaTeX names by appearance, rather than meaning
% By now, I've learned the standard LaTeX names, but I remember they used to give me trouble, so here are some macros
\newcommand{\isom}{\cong} %The isomorphism symbol
\newcommand{\union}{\cup}
\newcommand{\intersection}{\cap}
\newcommand{\bigunion}{\bigcup}
\newcommand{\bigintersection}{\bigcap}
\newcommand{\disjointunion}{\sqcup}
\newcommand{\bigdisjointunion}{\bigsqcup}
\newcommand{\hilight}[1]{\colorbox{yellow}{#1}}

%Some multiletter functions
\DeclareMathOperator{\Hom}{Hom}
\DeclareMathOperator{\Ext}{Ext}
\DeclareMathOperator{\End}{End}
\DeclareMathOperator{\Tor}{Tor}
\DeclareMathOperator{\Ker}{Ker}
\DeclareMathOperator{\CoKer}{CoKer}
\DeclareMathOperator{\Spec}{Spec}
\DeclareMathOperator{\Proj}{Proj}
\DeclareMathOperator{\Lie}{Lie}
\DeclareMathOperator{\Map}{Map}
\DeclareMathOperator{\Endo}{End}
\DeclareMathOperator{\tr}{tr}
\DeclareMathOperator{\ad}{ad}
\renewcommand{\Im}{\mathop{\mathrm{Im}}}
%Their calligraphic versions; use these for the sheaf constructions
\DeclareMathOperator{\HHom}{\mathcal{H} \textit{om}}
\DeclareMathOperator{\EExt}{\mathcal{E} \textit{xt}}
\DeclareMathOperator{\EEnd}{\mathcal{E} \textit{nd}}
\DeclareMathOperator{\TTor}{\mathcal{T} \textit{or}}
\DeclareMathOperator{\KKer}{\mathcal{K}\textit{er}}
\DeclareMathOperator{\CCoKer}{\mathcal{C} \textit{o}\mathcal{K} \textit{er}}
\newcommand{\IIm}{\mathop{\mathcal{I} \textit{m}}}
\newcommand{\ccH}{\mathscr{H}} %The very curly H

%This makes alternating tensors look right in displayed equations
\newcommand{\Alt}{\bigwedge\nolimits}

\newcommand{\mf}[1]{\mathfrak{#1}}

%Blackboard bold letters.
\renewcommand{\AA}{\mathbb{A}}
\newcommand{\BB}{\mathbb{B}}
\newcommand{\CC}{\mathbb{C}}
\newcommand{\DD}{\mathbb{D}}
\newcommand{\EE}{\mathbb{E}}
\newcommand{\FF}{\mathbb{F}}
\newcommand{\GG}{\mathbb{G}}
\newcommand{\HH}{\mathbb{H}}
\newcommand{\II}{\mathbb{I}}
\newcommand{\JJ}{\mathbb{J}}
\newcommand{\KK}{\mathbb{K}}
\newcommand{\LL}{\mathbb{L}}
\newcommand{\MM}{\mathbb{M}}
\newcommand{\NN}{\mathbb{N}}
\newcommand{\OO}{\mathbb{O}}
\newcommand{\PP}{\mathbb{P}}
\newcommand{\QQ}{\mathbb{Q}}
\newcommand{\RR}{\mathbb{R}}
\renewcommand{\SS}{\mathbb{S}}
\newcommand{\TT}{\mathbb{T}}
\newcommand{\UU}{\mathbb{U}}
\newcommand{\VV}{\mathbb{V}}
\newcommand{\WW}{\mathbb{W}}
\newcommand{\XX}{\mathbb{X}}
\newcommand{\YY}{\mathbb{Y}}
\newcommand{\ZZ}{\mathbb{Z}}

%Calligraphic letters

\newcommand{\cA}{\mathcal{A}}
\newcommand{\cB}{\mathcal{B}}
\newcommand{\cC}{\mathcal{C}}
\newcommand{\cD}{\mathcal{D}}
\newcommand{\cE}{\mathcal{E}}
\newcommand{\cF}{\mathcal{F}}
\newcommand{\cG}{\mathcal{G}}
\newcommand{\cH}{\mathcal{H}}
\newcommand{\cI}{\mathcal{I}}
\newcommand{\cJ}{\mathcal{J}}
\newcommand{\cK}{\mathcal{K}}
\newcommand{\cL}{\mathcal{L}}
\newcommand{\cM}{\mathcal{M}}
\newcommand{\cN}{\mathcal{N}}
\newcommand{\cO}{\mathcal{O}}
\newcommand{\cP}{\mathcal{P}}
\newcommand{\cQ}{\mathcal{Q}}
\newcommand{\cR}{\mathcal{R}}
\newcommand{\cS}{\mathcal{S}}
\newcommand{\cT}{\mathcal{T}}
\newcommand{\cU}{\mathcal{U}}
\newcommand{\cV}{\mathcal{V}}
\newcommand{\cW}{\mathcal{W}}
\newcommand{\cX}{\mathcal{X}}
\newcommand{\cY}{\mathcal{Y}}
\newcommand{\cZ}{\mathcal{Z}}
\newcommand{\fg}{\mathfrak{g}}
\newcommand{\fh}{\mathfrak{h}}
\newcommand{\fn}{\mathfrak{n}}
\newcommand{\fb}{\mathfrak{b}}
\newcommand{\GL}{\text{GL}}

\newtheorem{thm}{Theorem}
\newtheorem{lem}[thm]{Lemma}
\newtheorem{prop}[thm]{Proposition}
\newtheorem{cor}[thm]{Corollary}

\theoremstyle{definition}
\newtheorem{defn}[thm]{Definition}
\newtheorem{expl}[thm]{Example}
\newtheorem{expls}[thm]{Examples}
\newtheorem{rmk}[thm]{Remark}
\newtheorem{example}[thm]{Example}

\title{MAT5105 study notes}
\author{Nigel Redding}
\date{\today}

\begin{document}

\maketitle

% BERGE
\begin{thm} (Berge's theorem)
Let $M$ be a matching of $G$.
Then $M$ is maximal if and only if $G$ contains no $M$-augmenting path.
\end{thm}

\begin{proof}
$(\Rightarrow)$ Suppose $M$ is a maximal matching. Suppose for a contradiction that $G$ does indeed
contain an $M$-augmenting path $P = v_0 v_1 \ldots v_{2m} v_{2m+1}$. Then define
$$M' = (M \setminus \{v_1 v_2, v_3 v_4, \ldots, v_{2m-1} v_{2m}\}) 
       \cup \{v_0 v_1, \ldots, v_{2m} v_{2m+1}\}.$$
Then $|M'| = |M| - m + (m+1) = |M|+1$. Thus $M$ is not maximal. A contradiction.\\

$(\Leftarrow)$. Suppose $G$ does not contain an $M$-augmenting path. Then suppose for a contradiction
that $M$ is not maximal. Then there is a matching $M'$ such that $|M'| > |M|$. Then look at the graph
$H = G[M \triangle M']$. Each vertex in $H$ has degree at most $2$, obviously. Write
$H = \cup_{i=1}^k G_i$ where $G_1,\ldots,G_k$ are the connected components of $H$. Each $G_i$ is either
a path or a cycle. Pick one, $G_j$, that has more edges in $M'$ than $M$. One can easily get an $M$-augmnenting
path out of $G_j$.
\end{proof}

% TUTTE
\begin{thm} (Tutte's theorem)
$G$ has a perfect matching if and only if $o(G-S) \leq |S|$ for all $S \subseteq V$.
\end{thm}

\begin{proof}
Not on exam.
\end{proof}

% PETERSEN
\begin{thm} (Petersen's theorem)
If $G$ is a $3$-regular graph without cut edges, then $G$ has a perfect matching.
\end{thm}

\begin{proof}
We use Tutte's theorem. Let $S \subseteq V$. Let $G_1,\ldots,G_n$ be the odd components
of $G-S$. Let $V_i = V(G_i)$ and $E_i = E(G_i)$. Let $m_i$ be the number of edges with one end in 
$V_i$ and one end in $S$. Then $$\sum\limits_{v \in V_i} d_G(v_i) = 2|E_i| + m_i$$ and also
$$\sum\limits_{v \in V_i} d_G(v_i) = 3|V_i|.$$ These imply that $m_i$ is odd. Also, since $G$ has no cut edges,
$m_i \geq 3$. Hence 

\begin{align*}
	o(G-S) &= n \leq \dfrac{1}{3} \sum\limits_{i=1}^n m_i \\
	       &\leq \dfrac{1}{3} \sum\limits_{v \in S} d(v) \\
	       &= |S|
\end{align*}

and this completes the proof.
\end{proof}

% DIRAC
\begin{thm} (Dirac's theorem)
Let $G$ be a simple graph of order $n \geq 3$ and minimum degree $\delta \geq n/2$. Then $G$ is Hamiltonian.
\end{thm}

\begin{proof}
Suppose that there exists a graph $G$ of order $n \geq 3$ with $\delta \geq n/2$ which is not Hamiltonian.
We can assume such a graph is maximal. Then $G$ is not complete, as $n \geq 3$ (obvious Hamiltonian cycle).
So there exist verticies $u,v \in V$ such that $u$ is not adjacent to $v$. Then as $G$ is maximal,
$G+uv$ has a Hamilton cycle. Thus we get a Hamilton path $v_0 v_1 \ldots v_k$ in $G$, with $u = v_0$ and
$v = v_k$. Let $S = \{v_i : u v_{i+1} \in E\}$ and $T = \{v_i : v v_i \in E\}$. Then
if $S \cap T \neq \emptyset$, then we could find a Hamilton cycle in $G$, 
which we cannot do. Thus, $S \cap T = \emptyset$. So 
$$d(u) + d(v) = |S| + |T| = |S \cup T| + |S \cap T| < n$$ contradicting our hypothesis that $\delta \geq n/2$. 
\end{proof}

% BONDY-CHVATAL
\begin{lem} (Bondy-Chvatal Lemma)
Let $G$ be a simple graph of order $n$. Let $u$ and $v$ be non-adjacent verticies with
$d(u) + d(v) \geq n$. Then $G$ is Hamiltonian if and only if $G+uv$ is Hamiltonian.
\end{lem}

\begin{proof}
$(\Rightarrow)$. Suppose $G$ is Hamiltonian. Then surely $G+uv$ is as well. This is obvious.\\

$(\Leftarrow)$. Suppose $G+uv$ is Hamiltonian, but $G$ is not. 
Get a Hamilton path in $G$ as above. Set up $S$ and $T$
as above. Get $S \cap T = \emptyset$ as above. Get a contradiction, as above.
\end{proof}

\begin{thm} (Bondy-Chvatal theorem)
Let $G$ be a simple graph. Then $G$ is Hamiltonian if and only if the closure is Hamiltonian.
\end{thm}

\begin{proof}
Obviously, if $G$ is Hamiltonian, then so is the closure. Conversely, suppose the closure is Hamiltonian.
Let $e_1,\ldots,e_k$ be the edges added to $G$ to get the closure $c(G)$. Then for
$1 \leq i \leq k$, set $E_i = \{e_1,\ldots,e_k\}$. Then 

\begin{align*}
	G \text{ is Hamiltonian } & \iff G+E_1 \text{ is Hamiltonian } \\
				  & \iff G+E_2 \text{ is Hamiltonian } \\
				  & \ldots \\
				  & \iff G+E_{k-1} \text{ is Hamiltonian } \\
				  & \iff G+E_k = c(G) \text{ is Hamiltonian } 
\end{align*}

and the last is surely a true statement, making the first true as well.
\end{proof}

% Chvatal
\begin{thm}
Let $G$ be a simple graph of order $n \geq 3$ with degree sequence $(d_1,d_2,\ldots,d_n)$, where
$d_1 \leq d_2 \leq \ldots \leq d_n$. Suppose that there is no value of $m$ less than $n/2$ for which
$d_m \leq m$ and $d_{n-m} \leq n-m$. Then $G$ is Hamiltonian.
\end{thm}

\begin{proof}
Let $G$ satisfy the hypotheses of the theorem. By the Bondy-Chvatal theorem, it is enough to show that
the closure $c(G)$ is complete and hence Hamiltonian (from which it follows that $G$ is Hamiltonian).
For a vertex $v$ of $c(G)$, we denote its degree by $d'(v)$.

Assume for a contradiction that $c(G)$ is not complete. Let $u$ and $v$ be non-adjacent verticies of
$c(G)$ with $d'(u) \leq d'(v)$, and $d'(u) + d'(v)$ as large as possible. Since we are working in the
closure, we must have $d'(u) + d'(v) < n$.

Now let $S$ be the set of verticies in $V \setminus \{v\}$ which are non-adjacent to $v$ in $c(G)$.
Let $T$ be the set of verticies in $V \setminus \{u\}$ which are non-adjacent to $u$ in $c(G)$. It is
clear that $$|S| = n - 1 - d'(v) \text{ and } |T| = n - 1 - d'(u).$$ By the choice of $u$ and $v$,
every vertex in $S$ has degree at most $d'(u)$, and every vertex in $T \cup \{u\}$ has degree
at most $d'(v)$. Setting $d'(u) = m$, we find that $c(G)$ has at least $m$ verticies of
degree at most $m$ and at least $n-m$ verticies of degree less than $n-m$. 
\end{proof}

% HALL
\begin{thm} (Hall's theorem)
Let $G$ be a bipartite graph with bipartition $(X,Y)$. Then $G$ contains a matching that saturates
every vertex in $X$ if and only if $|N(S)| \geq |S|$ for all $S \subseteq X$.
\end{thm}

\begin{proof}
$(\Rightarrow).$ Obvious.\\

$(\Leftarrow).$ Suppose $|N(S)| \geq |S|$ for all $S \subseteq X$.
Let $m = |X|$. We proceed by induction on $m$. When $m=1$, the theorem is obvious. Assume it holds
for $k \in \{1,\ldots,m-1\}$.  There are two cases.\\

\emph{Case 1.} Suppose $|N(S)| \geq |S|+1$ for all $S \subseteq V_1$. Then let $e = v_1 v_2 \in E$ be an 
arbitrary edge. Then let $G' = G \setminus \{v_1,v_2\}$. Clearly $G'$ satisfies Hall's condition, and so
by induction it has a matching saturating $V_1$. Complete the matching in the obvious way.\\

\emph{Case 2.} Suppose $|N(T)| = |T|$ for some $T \subseteq V_1$. Then let 
$G' = G[T \cup N(T)]$ and $G'' = G[(V_1 \setminus T) \cup (V_2 \setminus N(T)]$. Both of these satisfy Hall's conditon
and thus both have matchings by induction, the first saturating $T$ and the second saturating $V_1 \setminus T$. 
Merge these two to get a matching of $G$ saturating $V_1$.

\end{proof}

% MARRIAGE
\begin{thm} (Hall's Marriage theorem)
If $G$ is a $k$-regular bipartite graph with $k > 0$, then $G$ has a perfect matching.
\end{thm}

\begin{proof}
Let $S \subseteq X$. By Hall's theorem it is enough to show that 
$|N(S)| \geq |S|$. Let $E_1$ be the set of edges incident with $S$, and let 
$E_2$ be the set of edges incident with $N(S)$. By definition,
$E_1 \subseteq E_2$. Thus, $$k |N(S)| = |E_2| \geq |E_1| = k|S|$$ which implies $|N(S)| \geq |S|$.
\end{proof}

% VIZING
\begin{thm} (Vizing's theorem)
$\Delta \leq \chi' \leq \Delta+1$.
\end{thm}

\begin{proof}
Not on exam.
\end{proof}

\begin{lem}
Suppose $G$ is not an odd cycle. Then $G$ has a $2$-edge colouring in which both colours are 
represented at each vertex of degree at least $2$.
\end{lem}

\begin{proof}
We have two cases. First, suppose $G$ is Eulerian. If $G$ is an even cycle, then the claim is obvious.
Suppose that $G$ is not an even cycle. Then, as $G$ is Eulerian, each vertex has degree $\geq 2$.
But since $G$ is not an even cycle, some vertex $v_0$ must have degree $\geq 4$. Let our euler tour
$P$ be given by $P = v_0 e_1 v_1 e_2 \ldots e_k v_k e_{k+1} v_0$. Then define
$E_1 = \{e_i : i \text{ even}\}$ and $E_2 = \{e_i : i \text{ odd}\}$. Clearly
$(E_1,E_2)$ is a $2$-edge colouring of $G$ and has the desired properties.\\

Now suppose $G$ is not Eulerian. Add a ``dummy'' vertex $v_0$ to $G$, and add an edge between $v_0$
and all vertecies of odd degree. Call the resulting graph $G'$. Obviously $G'$ is indeed Eulerian,
and thus has an euler tour $v_0 e_1 \ldots e_k v_k e_{k+1} v_0$. Then define $(E_1,E_2)$ as above. Then
$(E_1 \cap E, E_2 \cap E)$ is a $2$-colouring with the desired properties.
\end{proof}

\begin{lem}
Let $\cC = (E_1,\ldots,E_k)$ be an optimal $k$-edge colouring of $G$. 
If there is a vertex $u$ in $G$ and colours $i$ and $j$ such that $i$ is not represented at 
$u$, but $j$ is represented at $u$ at least twice, then the component
of $G[E_i \cup E_j]$ containing $u$ is an odd cycle.
\end{lem}

\begin{proof}
Suppose the component of $G[E_i \cup E_j]$ containing $u$, call it $U$, is \emph{not} an odd cycle.
Then by the last theorem, there is a $2$-edge colouring $\cD$ of $U$ in which both colours are represented
at $u$. Then recolour the edges of $H$ using $\cD$ and colour all the other edges with $\cC$. This new colouring,
$\cC'$, is such that $$\sum c'(v) > \sum c(v)$$ which shows that $\cC$ is not optimal. A contradiction.
\end{proof}

\begin{thm}
If $G$ is bipartite, then $\chi' = \Delta$.
\end{thm}

\begin{proof}
Well, obviously $\chi' \geq \Delta$. Suppose for a contradiction
that $\chi' > \Delta$. Let $\cC = (E_1,\ldots,E_\Delta)$ be an optimal 
$\Delta$-edge colouring of $G$. Then there must be some vertex $u$ such that
$c(u) < d(u)$. So, there are colours $i,j$ of $\cC$ such that $i$ is not represented
at $u$, but $j$ is represented at least twice at $u$. Then the component of $G[E_i \cup E_j]$ 
containing $u$ is an odd cycle. But that means $G$ is not bipartite. A contradiction.
\end{proof}

\begin{prop}
$S \subseteq V$ is independent if and only if $V \setminus S$ is a covering.
\end{prop}

\begin{proof}
Suppose $S$ is independent. Then let $e = xy \in E$. It is not possible for both $x$ and
$y$ to be members of $S$. Thus, at least one of $x$ or $y$ is in $V \setminus S$. So every edge
of $G$ has at least one end in $V \setminus S$, so $V \setminus S$ is a covering.

Suppose $V \setminus S$ is a covering. Then let $x$ and $y$ be distinct members of $S$. Then one
cannot have $x \sim_G y$, since then we would have an edge with both ends in $S$, contradicting the
hypothesis that $V \setminus S$ is a covering.
\end{proof}

Define $\alpha$ to be the maximum cardinality of an indpendent set and define $\beta$ to be the
minimum cardinality of a covering.

\begin{prop}
$\alpha + \beta = |V|$.
\end{prop}

\begin{proof}
Let $A$ be a maximum indepdendent set and let $B$ be a mimimum covering. Then

\begin{align*}
	|V| - \alpha &= |V \setminus A| \geq \beta \\
	|V| - \beta  &= |V \setminus B| \leq \alpha \\
\end{align*}

And thus both relations $|V| \geq \alpha + \beta$ and $|V| \leq \alpha + \beta$ hold.
Thus $|V| = \alpha + \beta$.
\end{proof}

\end{document}

