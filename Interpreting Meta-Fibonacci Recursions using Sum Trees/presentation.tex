% $Header$

\documentclass{beamer}

\mode<presentation>
{
  \usetheme{Warsaw}
  % or ...

  \setbeamercovered{transparent}
  % or whatever (possibly just delete it)
}

\usepackage[english]{babel}
\usepackage[latin1]{inputenc}
\usepackage{times}
\usepackage[T1]{fontenc}
% Or whatever. Note that the encoding and the font should match.
% If T1 does not look nice, try deleting the line with the fontenc.

\usepackage{epstopdf}
% for pictures
\usepackage{booktabs}

\title{Interpreting Meta-Fibonacci Recursions using Sum Trees}

\subtitle{An Overview}

\author{Robert Lech}
% - Use the \inst{?} command only if the authors have different
%   affiliation.

\institute[Universities of Somewhere and Elsewhere] % (optional, but mostly needed)
{
  Department of Mathematics\\
  Carleton University
}
% - Use the \inst command only if there are several affiliations.
% - Keep it simple, no one is interested in your street address.

\date[Short Occasion] % (optional)
{\today~/~Informal Presentation}

\subject{Talks}
% This is only inserted into the PDF information catalog. Can be left
% out.



% If you have a file called "university-logo-filename.xxx", where xxx
% is a graphic format that can be processed by latex or pdflatex,
% resp., then you can add a logo as follows:

% \pgfdeclareimage[height=0.5cm]{university-logo}{university-logo-filename}
% \logo{\pgfuseimage{university-logo}}

\AtBeginSubsection[]
{
  \begin{frame}<beamer>[allowframebreaks]{Outline}
    \tableofcontents[currentsection,currentsubsection]
  \end{frame}
}


% If you wish to uncover everything in a step-wise fashion, uncomment
% the following command:

%\beamerdefaultoverlayspecification{<+->}


\begin{document}

\begin{frame}
  \titlepage{}
\end{frame}

\begin{frame}[allowframebreaks]{Outline}
  \tableofcontents
\end{frame}

\section{Introduction}

\subsection{Motivation}

\begin{frame}{What are Meta-Fibonacci recursions?}

\begin{enumerate}
  \item Example of linear recurrence
  \begin{enumerate}
    \item $F(n)=F(n-1)+F(n-2)$, $F(1)=F(2)=1$
    \pause{}
  \end{enumerate}
  \item Meta-Fibonacci recursions are nested recurrence relations that look similar to linear recursions
  \begin{enumerate}
    \item $R(n) = R(n-R(n-1)) + 1$, $R(1)=1$ (Golumb's Recursion)
    \pause{}
    \item $G(n) = n-G(G(n-1))$, $G(1)=1$ (Hofstadter's G-Recursion)
    \pause{}
    \item $A(n) = A(n-A(n-1)) + A(A(n-1)), A(1)=A(2)=1$ (Conway's Recursion)
    \pause{}
    \item $V(n) = V(n-V(n-1)) + V(n-V(n-4))$, $V(1)=1, V(2)=2, V(3)=3, V(4)=4$ (V-Recursion)
    \pause{}
    \item $C(n) = C(n-C(n-1)) + C(n-1-C(n-2)), C(1)=C(2)=1$ (Conolly Recursion)
    \pause{}
    \item $Q(n) = Q(n-Q(n-1)) + Q(n-Q(n-2)), Q(1)=Q(2)=1$ (Hofstadter's Q-Recursion)
  \end{enumerate}
\end{enumerate}

\end{frame}

\begin{frame}{Example: How to Simplify}

Let's evaluate $G(14)$ by telescoping outwards. That is, if we let $G(1)=1$, and we evaluated $G(2), G(3), \ldots, G(13)$  then we can create the table below.
\begin{table}[h]
\centering
\begin{tabular}{@{}c*{13}{c}@{}}
  \toprule
  $n$ 		& 1 & 2 & 3 & 4 & 5 & 6 & 7 & 8 & 9 & 10 & 11 & 12 & 13 \\
  \midrule
  $G(n)$ 	& 1 & 1 & 2 & 3 & 3 & 4 & 4 & 5 & 6 &  6 &  7	&  8 &  8 \\
  \bottomrule
\end{tabular}
\caption{The first few terms in the output of various recursions.}
\label{tab:recursions_fewterms}
\end{table}
\begin{align*}
G(14)
&= 14-G(G(13))\\
\pause{}
&= 14-G(8)\\
\pause{}
&= 14-8+G(G(7))\\
\pause{}
&= 14-8+G(4)\\
\end{align*}

\end{frame}

\begin{frame}{Example: How to Simplify}

\begin{align*}
G(14)
&= 14-8+G(4)\\
\pause{}
&= \ldots\\
\pause{}
&= 14-8+4-2+G(1)\\
\pause{}
&= 14-8+4-2+1\\
\pause{}
&= 9
\end{align*}

Therefore, $G(14)=9$

\end{frame}

\begin{frame}{How do they behave?}

Similarly, we can calculate the the values for various recursions. Note that not all of them are `slow-growing' (successive terms increase by 0 or 1).

\begin{table}[h]
\centering
\begin{tabular}{@{}c*{12}{c}@{}}
  \toprule
  $n$ 		& 1 & 2 & 3 & 4 & 5 & 6 & \ldots & 33 & 34 & 35 & 36 & 37	\\
  \midrule
  $R(n)$ 	& 1 & 2 & 2 & 3 & 3 & 3 & \ldots & 8   & 8 & 8 & 8 & 9  		\\
  $G(n)$ 	& 1 & 1 & 2 & 3 & 3 & 4 & \ldots & 21 & 21 & 22 & 22 & 23 	\\
  $A(n)$ 	& 1 & 1 & 2 & 2 & 3 & 4 & \ldots & 17 & 18 & 19 & 20 & 21 	\\
  $V(n)$		& 1 & 2 & 3 & 4 & 5 & 5 & \ldots & 20 & 21 & 21 & 22 & 22 	\\
  $C(n)$		& 1 & 1 & 2 & 2 & 3 & 4 & \ldots & 17 & 18 & 18 & 19 & 20 	\\
  $Q(n)$		& 1 & 1 & 2 & 3 & 3 & 4 & \ldots & 17 & 20 & 21 & 19 & 20	\\
  \bottomrule
\end{tabular}
\caption{The first few terms in the output of various recursions.}
\label{tab:recursions_fewterms}
\end{table}

\end{frame}

\begin{frame}{Why study them?}

\begin{enumerate}
  \item We don't know how to predict their behaviour
  \begin{enumerate}
    \item Recursions are sensitive to initial conditions and parameters
    \pause{}
  \end{enumerate}
  \item We don't know how to solve them generally
  \begin{enumerate}
    \item Solutions are complex and ad-hoc
    \pause{}
  \end{enumerate}
  \item We'd like to know if there's even a pattern worth investigating
  \begin{enumerate}
    \item What we'll look for in this presentation
  \end{enumerate}
\end{enumerate}

\end{frame}

\begin{frame}{Tree Interpretations}

\begin{figure}[htbp]
\center{
\includegraphics[scale=0.35]{images/C(n)=C(n-2-C(n-1))+C(n-3-C(n-2)).pdf} % chktex 36 - ignore parentheses in filenames
\caption{Tree interpretation of $C_2(n)=C_2(n-2-C_2(n-1))+C_2(n-3-C_2(n-2))$ with initial conditions 1, 1, 1, 2. Counting the number of leaves with labels less than or equal to $n$ returns $C_2(n)$. As an example, $C_2(8)=3$, $C_2(13)=4$}
\label{fig:C_s=2_j=1_IC1112_35terms}
}
\end{figure}

\end{frame}

\begin{frame}{Tree Interpretations}

\begin{figure}[htbp]
\center{
\includegraphics[scale=0.7]{images/lazy-golomb.png}
\caption{The tree $G_2$ up to label 13. The $s$-nodes are drawn as squares and the $j$-nodes as circles.}
\label{fig:C_s=2_j=1_IC1112_35terms}
}
\end{figure}

\end{frame}

\subsection{Understanding Recursions}

\begin{frame}{Let's try to define what we're looking at}

\begin{definition}
\label{general_recursion}
We loosely define an \emph{arity-1, non-nested recursion} as something that looks like $R(n)=cR(Q(n))+P(n)$ where $P(n)$ and $Q(n)$ are polynomials over the naturals and $1\leq Q(n)<n$ INDEPENDENT of $R(n)$.
\end{definition}
\pause{}
\begin{enumerate}
  \item Examples
  \begin{enumerate}
    \item $R(n)=R(n-1)+3$, $R(1)=1$
    \item $R(n)=n-R(n-1)$, $R(1)=1$
  \end{enumerate}
\end{enumerate}

\end{frame}

\begin{frame}{Let's try to define what we're looking at}

\begin{definition}
\label{general_recursion}
We loosely define an \emph{arity-1, nested recursion} as something that looks like $R(n)=cR(Q(n))+P(n)$ where $P(n)$ and $Q(n)$ are polynomials over the naturals and $1\leq Q(n)<n$ where $Q(n)$ depends on $R(n)$.
\end{definition}
\pause{}
\begin{enumerate}
  \item Examples
  \begin{enumerate}
    \item $G(n)=n-G(G(n-1))$, $G(1)=1$
    \item $R(n) = R(n-R(n-1))+1, R(1)=1$
  \end{enumerate}
\end{enumerate}

\end{frame}

\section{Generating Sum Trees for Arity-1 Recursions}

\subsection{Method of Generating PSTs}

\begin{frame}{The Algorithm for Creating a  Path-Sum Tree-Interpretation}

\begin{enumerate}
  \item How can we come up with a tree for an arity-1 recursion of the type we defined?
  \begin{enumerate}
    \item Step 1: Telescope outwards to find the sum of the terms.
    \pause{}
    \item Step 2: Interpret this decreasing sequence as a labelled path of vertices
    \pause{}
    \item Step 3: Union these paths to create a labelled branching of a tree
  \end{enumerate}
\end{enumerate}

\end{frame}

\begin{frame}{Example: How to Simplify}

\begin{enumerate}
  \item Step 1: Let's recall a previous example, and compute a few others out of interest.
\end{enumerate}
\[
\begin{aligned}
G(12)&=12-7+4-2+1\\
G(13)&=13-8+4-2+1\\
G(14)&=14-8+4-2+1\\
G(15)&=15-9+5-3+1\\
G(16)&=16-9+5-3+1\\
G(17)&=17-10+6-3+1
\end{aligned}
\]

In fact, we can note that:

\end{frame}

\begin{frame}{Defining $g(n)$ and $r(n)$}

\begin{enumerate}
  \item Keeping $G(n)=n-G(G(n-1))$ in mind, let's define $g(n)$ to be the alternating sum along a path of vertices.
  \item Keeping $R(n)=R(n-R(n-1))+1$ in mind, let's define $r(n)$ to be the length of the path.
\end{enumerate}

\end{frame}

\begin{frame}{Step 1 and 2: Creating the Path for $G(n)$}

\begin{figure}[h]
\centerline{\includegraphics[scale=0.25]{images/G_of_14_PST_gen_algo.png}}
\caption{We can calculate $g(14)$ by taking the alternating sum over the nodes. The alternating sum over the last column of nodes gives us that $g(14)=9$}
\label{fig:calculating_g14}
\end{figure}

\end{frame}

\begin{frame}{Step 1 and 2: Creating the Path for $G(n)$}

\begin{figure}[h]
\centerline{\includegraphics[scale=0.25]{images/G_of_12_13_14_15_16_17_paths.png}}
\caption{We can create paths similarly for many different nodes. Note how the paths can eventually converge\ldots}
\label{fig:calculating_paths_g12_g13_g14_g15_g16_g17}
\end{figure}

\end{frame}

\begin{frame}{Step 3: Unioning of Paths for $G(n)$}

\begin{figure}[h]
\centerline{\includegraphics[scale=0.3]{images/Path_recursion_of_G_2_of_12_and_14.png}}
\caption{A unioning of the paths for $G(12)$ and $G(14)$}
\label{fig:G2example}
\end{figure}

\end{frame}

\begin{frame}{Step 3: The General Tree for $G(n)$}

\begin{figure}[h]
\centerline{\includegraphics[scale=0.26]{images/G_Tree.png}}
\caption{A tree-interpretation for $G(n)$}
\label{fig:g2_tree}
\end{figure}

\end{frame}

\begin{frame}{Another Tree: $R(n)$}

\begin{figure}[h]
\centerline{\includegraphics[scale=0.28]{images/Golumb_j=1_Tree.png}}
\caption{The recursion for $R(n)$ up to depth $d=5$.}
\label{fig:R(n)}
\end{figure}

\end{frame}

\section{Understanding the j- and k-parameters}

\subsection{Introduction}

\begin{frame}{Parametrized Recursions}

\begin{enumerate}
  \item Often we talk about a ``family'' of nested recursions, algebraically related by an introduced parameters
  \pause{}
  \begin{enumerate}
    \item $V_j(n)=V_j(n-V_j(n-j))+V_j(n-V_j(n-4j))$, $\forall i \in \{1,\ldots,9\}, V_j(i)=i$
    \pause{}
    \item $G_k(n)=n-G^k(n-1)$, $G(1)=1$ where $k$ denotes the $k^{th}$ composition of $G$
    \pause{}
    \begin{enumerate}
      \item We note that: $G_2(n)=n-G_2(G_2(n-1))$
      \pause{}
    \end{enumerate}
  \end{enumerate}
  \item Although there exists a relationship between $V(n)=V_1(n)$ and $V_j(n)$, we don't exactly know the relationship (or whether there is one) between $G_1(n)$ or $G_2(n)$ and $G_k(n)$.
  \pause{}
  \begin{enumerate}
    \item In this presentation, we'll focus on the lesser understood $k$-parameter.
  \end{enumerate}
\end{enumerate}

\end{frame}

\subsection{Understanding the k-parameter}

\begin{frame}{Tree for $G_1(n)$}

\begin{figure}[h]
\centerline{\includegraphics[scale=0.25]{images/G1(n).png}}
\caption{The recursion for $G_1(n)=n-G_1(n-1)$ up to depth $d=5$.}
\label{fig:G(n)}
\end{figure}

\end{frame}

\begin{frame}{Recall Tree for $G_2(n)$}

\begin{figure}[h]
\centerline{\includegraphics[scale=0.26]{images/G_2_Tree.png}}
\caption{A tree-interpretation for $G_2(n)$}
\label{fig:g2_tree}
\end{figure}

\end{frame}

\begin{frame}{Recursive Definition for $G_2(n)$}

\begin{figure}[h]
\centerline{\includegraphics[scale=0.35]{images/G_2_Recursion.jpg}}
\caption{The visualized pattern for $G_k$ where $k=2$ with ICs 1.}
\label{fig:G2}
\end{figure}

\end{frame}

\begin{frame}{Tree for $G_3(n)$}

\begin{figure}[h]
\centerline{\includegraphics[scale=0.26]{images/Gk3_ICs1_4levels.png}}
\caption{The recursion for $G_{k}$ when $k=3$ with ICs 1.}
\label{fig:Gk3_ICs1_4levels}
\end{figure}

\end{frame}

\begin{frame}{Recursive Definition for $G_2(n)$}

\begin{figure}[h]
\centerline{\includegraphics[scale=0.27]{images/G_3_Recursion.jpg}}
\caption{The visualized pattern for $G_k$ where $k=3$ with ICs 1.}
\label{fig:G3}
\end{figure}

\end{frame}

\begin{frame}{Recursive Definition for $G_k(n)$}

\begin{figure}[h]
\centerline{\includegraphics[scale=0.35]{images/G_k_recursion.png}}
\caption{The recursion for $G_k$.}
\label{fig:Gk}
\end{figure}

\end{frame}

\begin{frame}{Connecting $G_1$ to $G_k$, $k>1$}

\begin{enumerate}
  \item We know the solutions for $G_k(n)$ for some values of $k$
  \begin{enumerate}
    \item $G_1(n)=\lfloor\frac{n+1}{2}\rfloor$
    \item $G_2(n)=\lfloor\frac{n+1}{\phi}\rfloor$
  \end{enumerate}
  \pause{}
  \item We don't know them for values of $k>2$
  \item Since $\phi$ is related to the width of the tree for $G_2(n)$, maybe it's worth looking at the widths of trees for $G_k(n)$, $k>2$
\end{enumerate}

\end{frame}

\begin{frame}{Estimating the width of a tree}

\begin{table}[h]
\centering
\begin{tabular}{@{}l*{5}{c}@{}}
  \toprule
		& $k=1$	& $k=2$	& $k=3$	& $k=4$	& $k=5$	\\
  \midrule
  row 1	&	1	&	1	 & 1	&	1		   & 1 \\
  row 2	&	1	&	2	 & 3	&	4		   & 5 \\
  row 3	&	1	&	3	 & 6	&	10		 & ? \\ % chktex 26 - ignore spacing before `?`
  row 4	&	1	&	5	 & 13	&	26		 & ? \\ % chktex 26 - ...
  row 5	&	1	&	8	 & 28	&	69     & ? \\ % chktex 26 - ...
  row 6	&	1	&	13 & 60	& (181?) & ? \\ % chktex 26 - ...
  \bottomrule
\end{tabular}
\caption{The table cells indicate how many nodes in the $i^{th}$ row of the PST at $G_{k}(n)$ for various $k$.}
\label{table:rowCount_table}
\end{table}

\end{frame}

\begin{frame}{Estimating the width of a tree Formula}

\begin{enumerate}
  \item Let $RC_{k,h}$ denote the width of tree $G_k(n)$ at height $h$. Though $RC_{k,h}$ may not be immediately obvious, but if you note that:
  \begin{enumerate}
    \item $RC_{1,h}=RC_{1,h-1}$, $RC_{1,1}=1$
    \item $RC_{2,h}=RC_{2,h-1}+RC_{2,h-2}$, $RC_{2,1}=1$, $RC_{2,2}=2$
  \end{enumerate}
  \item Then we may guess that:
  \begin{enumerate}
    \item $RC_{3,h}=RC_{3,h-1}+2RC_{3,h-2}+RC_{3,h-3}$, $RC_{3,1}=1$, $RC_{3,2}=3$, $RC_{3,3}=6$
    \item $RC_{4,h}=RC_{4,h-1}+3RC_{4,h-2}+3RC_{4,h-3}+RC_{4,h-4}$, $RC_{4,1}=1$, $RC_{4,2}=4$, $RC_{4,3}=10$, $RC_{4,4}=24$
  \end{enumerate}
\end{enumerate}

\end{frame}

\begin{frame}{Estimating the width of a tree}

So we're being asked to solve linear recursions of the form:

\[
  RC_{k,h} = \sum_{i=0}^{k-1} {k-1 \choose i} RC_{k,h-i-1}
\]

Which implies we need to solve the OGF below:

\[
  RC_{k}(z)=\frac{\ldots}{1-z(1+z)^{k-1}}
\]

If we look at asymptotics:

\begin{table}[h]
\centering
\begin{tabular}{@{}l*{4}{c}@{}}
  \toprule
		& $k=1$	& $k=2$	& $k=3$	& $k=4$		\\
  \midrule
  $\lim_{h\to\infty}\frac{RC_{k,h}}{RC_{k,h-1}}$	&	1	&	1.61803	&	2.1479	&	2.62966	\\
  \bottomrule
\end{tabular}
\caption{The asymptotic value of $RC_{k,h}/RC_{k,h-1}
$ as $h\to\infty$}
\label{table:rowCount_table}
\end{table}

\end{frame}

\begin{frame}{Connecting $G_1$ to $G_k$, $k>1$ (revisited)}

\begin{enumerate}
  \item This all being said, $G_3(n)\ne \lfloor\frac{n+1}{2.1479}\rfloor$
  \pause{}
  \item More research needs to go into this \ldots
\end{enumerate}

\end{frame}

\section{Incomplete/Future Directions}

\begin{frame}{Incomplete/Future Directions}

\begin{enumerate}
  \item Understanding Non-Computable Recursions
  \begin{enumerate}
    \item We saw equivalent recursions for $R(n)$ that were non-computable
  \end{enumerate}
  \item Generating Sum Trees for Arity 2 or more
  \begin{enumerate}
    \item Is there a way to have interpretations that aren't messy/tangled?
  \end{enumerate}
  \item Isgur's Trees and how they relate to Path Sum Trees
  \item Applications of recursive trees
\end{enumerate}

\end{frame}

\end{document}
