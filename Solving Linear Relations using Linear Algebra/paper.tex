% Robert Lech - June 2015
 
\documentclass{article}
\usepackage{times}
\usepackage{color}
\usepackage{amsmath}
 
\begin{document}
 
% Article top matter
\title{How to Solve the Fibonacci Recursion using Linear Algebra} %\LaTeX is a macro for printing the Latex logo
\author{Robert Lech\\
		Carleton University,\\
		Ottawa, Canada,\\
		K1S 5B6\\
		\texttt{robert.lech123@gmail.com}}  %\texttt formats the text to a typewriter style font
\date{\today}  %\today is replaced with the current date
\maketitle
 
\begin{abstract}
We're curious to understand why the Fibonacci recursion $R(n)=R(n-1)+R(n-2), R(0)=R(1)=1$ has the closed form solution $R(n)=C_{1}\lambda_{1}^{n}+C_{2}\lambda_{2}^{n}$ where $C_{1}=\frac{1}{\sqrt{5}}$ and $C_{2}=-\frac{1}{\sqrt{5}}$.
In this paper, we'll see how to solve this recursion using Linear Algebra techniques. 

\textcolor{red}{FINISH THIS}
\end{abstract}
 
\section{Introduction}
\textcolor{red}{FINISH THIS}
 
\section{Stuff}
We're interested in solving:

\begin{equation}
\label{eqn:fib_rec}
R(n)=R(n-1)+R(n-2), R(0)=0, R(1)=1
\footnote{The reader can note that solving this recursion with the initial 
conditions $R(0)=0, R(1)=1$ or doing so with the intial conditions % chktex 36
$R(1)=1, R(2)=1$ gives us the same closed form solution as they output the % chktex 36
same sequence. We use the former set of initial conditions because it allows 
for less computations.}
\end{equation}

We first want to rewrite this problem as a Linear Algebra problem. To do so, 
we'll introduce a concept called the ``Companion Matrix''. Its purpose is to 
transform a vector of sequence elements 
$\begin{bmatrix} y_{n} \\ y_{n-1} \end{bmatrix}$
to $\begin{bmatrix} y_{n+1} \\ y_{n} \end{bmatrix}$. 
Therefore, we'll say that:

\begin{align*} 
\vec{y_{n}}
=
\begin{bmatrix}
y_{n+1} \\
y_{n} 
\end{bmatrix}
&=
\begin{bmatrix}
1 & 1 \\
1 & 0 
\end{bmatrix}
\begin{bmatrix}
y_{n} \\
y_{n-1}
\end{bmatrix} \\
&=
\vec{C}
\begin{bmatrix}
y_{n} \\
y_{n-1} 
\end{bmatrix} \\
&=  
\vec{C}\vec{C}
\begin{bmatrix}
y_{n-1} \\
y_{n-2} 
\end{bmatrix} \\ 
&=  
\vec{C}^{2}
\begin{bmatrix}
y_{n-1} \\
y_{n-2} 
\end{bmatrix} \\ 
\vdots\\
&=  
\vec{C}^{n}
\begin{bmatrix}
y_{1} \\
y_{0} 
\end{bmatrix}
\end{align*}

Therefore, to solve $\vec{y_{n}}$, we'd need to solve $\vec{C}^{n}$. There is 
no immediate answer to this as computing the general $n^{th}$ power of a 
general matrix can get quite ugly. However, we can recall that calculating the 
$n^{th}$ power of a diagonalizable matrix is a lot easier. Let $\lambda_{i}$ 
and $\vec{v_i}$ be $i^{th}$ eigenvalue and eigenvector of $\vec{C}$, 
respectively. Then we can let $\vec{P}=(\vec{v_i})^{T}$ be the matrix create 
by the eigenvectors of $\vec{C}$ and we can let $\vec{\Lambda}$ be the matrix 
who's diagonal entries are the corresponding $\lambda$ values. 
We can also use Gauss-Jordan Elimination to compute $\vec{P}^{-1}$. We can 
recall that:

\begin{equation}
\vec{C}^{n}
=
\left(\vec{P}\vec{\Lambda}\vec{P}^{-1}\right)^{n}
=
\vec{P}\vec{\Lambda}^{n}\vec{P}^{-1}
\end{equation}

This is a lot easier to compute. We can obtain the eigenvalues 
$\lambda_{1}=\frac{1-\sqrt{5}}{2}$ and $\lambda_{2}=\frac{1+\sqrt{5}}{2}$. 
The respective eigenvectors are 
$\vec{v_1}= \begin{bmatrix}\lambda_{1} \\ 1 \end{bmatrix}$ and 
$\vec{v_2} = \begin{bmatrix}\lambda_{2} \\ 1 \end{bmatrix}$. 
Therefore, we have that:

\begin{align*}
\begin{bmatrix}
1 & 1 \\
1 & 0 
\end{bmatrix}^{n}
&=
\left(
\begin{bmatrix}
\lambda_{1} & \lambda_{2} \\
1 & 1 
\end{bmatrix}
\begin{bmatrix}
\lambda_{1} & 0 \\
0 & \lambda_{2} 
\end{bmatrix}
\begin{bmatrix}
-(\lambda_{2}-\lambda_{1})^{-1} & 1+(\lambda_{1})(\lambda_{2}-\lambda_{1})^{-1} \\
(\lambda_{2}-\lambda_{1})^{-1} & -(\lambda_{1})(\lambda_{2}-\lambda_{1})^{-1} 
\end{bmatrix}
\right)^{n} \\
&=
\begin{bmatrix}
\lambda_{1} & \lambda_{2} \\
1 & 1 
\end{bmatrix}
\begin{bmatrix}
\lambda_{1} & 0 \\
0 & \lambda_{2} 
\end{bmatrix}^{n}
\begin{bmatrix}
-(\lambda_{2}-\lambda_{1})^{-1} & 1+(\lambda_{1})(\lambda_{2}-\lambda_{1})^{-1} \\
(\lambda_{2}-\lambda_{1})^{-1} & -(\lambda_{1})(\lambda_{2}-\lambda_{1})^{-1} 
\end{bmatrix}\\
&=
\begin{bmatrix}
\lambda_{1} & \lambda_{2} \\
1 & 1 
\end{bmatrix}
\begin{bmatrix}
\lambda_{1}^{n} & 0 \\
0 & \lambda_{2}^{n}
\end{bmatrix}
\begin{bmatrix}
-(\lambda_{2}-\lambda_{1})^{-1} & 1+(\lambda_{1})(\lambda_{2}-\lambda_{1})^{-1} \\
(\lambda_{2}-\lambda_{1})^{-1} & -(\lambda_{1})(\lambda_{2}-\lambda_{1})^{-1} 
\end{bmatrix}\\
&=
\begin{bmatrix}
\lambda_{1}^{n+1} & \lambda_{2}^{n+1} \\
\lambda_{1}^{n} & \lambda_{2}^{n}
\end{bmatrix}
\begin{bmatrix}
-(\lambda_{2}-\lambda_{1})^{-1} & 1+(\lambda_{1})(\lambda_{2}-\lambda_{1})^{-1} \\
(\lambda_{2}-\lambda_{1})^{-1} & -(\lambda_{1})(\lambda_{2}-\lambda_{1})^{-1} 
\end{bmatrix}\\
&=
\begin{bmatrix}
-\lambda_{1}^{n+1}(\lambda_{2}-\lambda_{1})^{-1} + \lambda_{2}^{n+1}(\lambda_{2}-\lambda_{1})^{-1} 
& \lambda_{1}^{n+1}(1+(\lambda_{1})(\lambda_{2}-\lambda_{1})^{-1}) - \lambda_{2}^{n+1}(\lambda_{1})(\lambda_{2}-\lambda_{1})^{-1} \\
-\lambda_{1}^{n}(\lambda_{2}-\lambda_{1})^{-1} + \lambda_{2}^{n}(\lambda_{2}-\lambda_{1})^{-1} 
& \lambda_{1}^{n}(1+(\lambda_{1})(\lambda_{2}-\lambda_{1})^{-1}) - \lambda_{2}^{n}(\lambda_{1})(\lambda_{2}-\lambda_{1})^{-1} \\
\end{bmatrix}\\
&=
\begin{bmatrix}
\frac{\lambda_{1}^{n+1} - \lambda_{2}^{n+1}}{\lambda_{1}-\lambda_{2}} 
& \lambda_{1}^{n+1}(\frac{\lambda_{2}}{\lambda_{2}-\lambda_{1}}) - \lambda_{2}^{n+1}(\frac{\lambda_{1}}{\lambda_{2}-\lambda_{1}}) \\
 & \\
\frac{\lambda_{1}^{n} - \lambda_{2}^{n}}{\lambda_{1}-\lambda_{2}} 
& \lambda_{1}^{n}(\frac{\lambda_{2}}{\lambda_{2}-\lambda_{1}}) - \lambda_{2}^{n}(\frac{\lambda_{1}}{\lambda_{2}-\lambda_{1}}) \\
\end{bmatrix}\\
&=
\begin{bmatrix}
\frac{\lambda_{1}^{n+1} - \lambda_{2}^{n+1}}{\lambda_{1}-\lambda_{2}} 
& \frac{\lambda_{1}^{n+1}\lambda_{2}-\lambda_{2}^{n+1}\lambda_{1}}{\lambda_{2}-\lambda_{1}}\\
 & \\
\frac{\lambda_{1}^{n} - \lambda_{2}^{n}}{\lambda_{1}-\lambda_{2}} 
& \frac{\lambda_{1}^{n}\lambda_{2}-\lambda_{2}^{n}\lambda_{1}}{\lambda_{2}-\lambda_{1}}\\
\end{bmatrix}\\
&=
\begin{bmatrix}
\frac{\lambda_{1}^{n+1} - \lambda_{2}^{n+1}}{\lambda_{1}-\lambda_{2}} 
& \frac{\lambda_{2}^{n+1}\lambda_{1}-\lambda_{1}^{n+1}\lambda_{2}}{\lambda_{1}-\lambda_{2}}\\
 & \\
\frac{\lambda_{1}^{n} - \lambda_{2}^{n}}{\lambda_{1}-\lambda_{2}} 
& \frac{\lambda_{2}^{n}\lambda_{1}-\lambda_{1}^{n}\lambda_{2}}{\lambda_{1}-\lambda_{2}}\\
\end{bmatrix}
\end{align*}

We note that $\lambda_{1}\lambda_{2}=-1$, and therefore the last step simplifies to:

\begin{align*}
\vec{C}^{n}
&=
\begin{bmatrix}
\frac{\lambda_{1}^{n+1} - \lambda_{2}^{n+1}}{\lambda_{1}-\lambda_{2}} 
& \frac{\lambda_{1}^{n} - \lambda_{2}^{n}}{\lambda_{1}-\lambda_{2}}\\
 & \\
\frac{\lambda_{1}^{n} - \lambda_{2}^{n}}{\lambda_{1}-\lambda_{2}} 
& \frac{\lambda_{1}^{n-1} - \lambda_{2}^{n-1}}{\lambda_{1}-\lambda_{2}} \\
\end{bmatrix}
\end{align*}

We can now simplify our previous steps to get that:

\begin{align*} 
\vec{y_{n}}
&=  
\begin{bmatrix}
y_{n+1} \\
y_{n} 
\end{bmatrix}\\
&=
\vec{C}^{n}
\begin{bmatrix}
y_{1} \\
y_{0} 
\end{bmatrix}\\
&=
\begin{bmatrix}
\frac{\lambda_{1}^{n+1} - \lambda_{2}^{n+1}}{\lambda_{1}-\lambda_{2}} 
& \frac{\lambda_{1}^{n} - \lambda_{2}^{n}}{\lambda_{1}-\lambda_{2}}\\
 & \\
\frac{\lambda_{1}^{n} - \lambda_{2}^{n}}{\lambda_{1}-\lambda_{2}} 
& \frac{\lambda_{1}^{n-1} - \lambda_{2}^{n-1}}{\lambda_{1}-\lambda_{2}} \\
\end{bmatrix}
\begin{bmatrix}
1 \\
0 
\end{bmatrix}\\
&=
\begin{bmatrix}
\frac{\lambda_{1}^{n+1} - \lambda_{2}^{n+1}}{\lambda_{1}-\lambda_{2}} \\
\\
\frac{\lambda_{1}^{n} - \lambda_{2}^{n}}{\lambda_{1}-\lambda_{2}}
\end{bmatrix}
\end{align*}

This implies that 
$y_{n}=\frac{\lambda_{1}^{n} - \lambda_{2}^{n}}{\lambda_{1}-\lambda_{2}}=\left(\frac{1}{\sqrt{5}}\right)\lambda_{1}^{n}+\left(-\frac{1}{\sqrt{5}}\right)\lambda_{2}^{n}$.
A similar argument can be applied to recursions of the form 
$R(n)=\sum_{i=1}^{k}c_{i}R(n-i)$ where $c_{i}$ is a constant.

\section{Limitations}
This method has limitations. It cannot be applied to recursions of the form 
$R(n)=\sum_{i=1}^{k}c_{i}R(n-i)$ where $c_{i}$ is not a constant. A
simple example would be $R(n)=nR(n-1), R(1)=1$. It is quite visible from 
telescoping that $R(n)=(n!)$. However, one would not be able to put this into
any simpler form using this method since computing 
$\vec{C}^n=\prod_{i=1}^{n}\begin{bmatrix}i\end{bmatrix}=n! $.

\begin{thebibliography}{9}
%The \bibitem is to start a new reference.  Ensure that the cite_key is
%unique.  You don't need to put each element on a new line, but I did
%simply for readability.

\textcolor{red}{WIKIPEDIA}

\end{thebibliography} %Must end the environment
 
\end{document}  %End of document.