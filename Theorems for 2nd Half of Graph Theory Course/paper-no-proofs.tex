\documentclass[12pt]{amsart}

\usepackage{enumerate, color, amssymb, amsmath, amsfonts, amsthm, graphics, mathrsfs}
%\usepackage[hmargin=1 in, vmargin = 0.5 in]{geometry}

%%%%%%% Pagestyle stuff %%%%%%%%%%%%%%%%%%%
 \usepackage{fancyhdr}                   %%
   \pagestyle{fancy}                     %%
   \fancyhf{} %delete the current section for header and footer
 \usepackage[paperheight=11in,%          %%
             paperwidth=8.5in,%          %%
             outer=1.2in,%               %%
             inner=1.2in,%               %%
             bottom=.7in,%               %%
             top=.7in,%                  %%
             includeheadfoot]{geometry}  %%
   \addtolength{\headwidth}{.75in}       %%
   \fancyhead[RO,LE]{\thepage}           %%
   \fancyhead[RE,LO]{\sectionname}       %%
   \setlength{\headheight}{15.8pt}       %%
   \raggedbottom{}                       %%
%%%%%%% End Pagestyle stuff %%%%%%%%%%%%%%%

% When you first define a new word, use this macro to make it stand out
% EG We say that an abelian group $I$ is \newword{injective} if, for any
% injection $G \to H$, and any map $G \to I$, there is a map $H \to I$ making the
% obvious diagram commute.
\newcommand{\newword}[1]{\textbf{\emph{#1}}}

%Arrows
\newcommand{\into}{\hookrightarrow}
\newcommand{\onto}{\twoheadrightarrow}

%Things LaTeX names by appearance, rather than meaning
% By now, I've learned the standard LaTeX names, but I remember they used to give me trouble, so here are some macros
\newcommand{\isom}{\cong} %The isomorphism symbol
\newcommand{\union}{\cup}
\newcommand{\intersection}{\cap}
\newcommand{\bigunion}{\bigcup}
\newcommand{\bigintersection}{\bigcap}
\newcommand{\disjointunion}{\sqcup}
\newcommand{\bigdisjointunion}{\bigsqcup}
\newcommand{\hilight}[1]{\colorbox{yellow}{#1}}

%Some multiletter functions
\DeclareMathOperator{\Hom}{Hom}
\DeclareMathOperator{\Ext}{Ext}
\DeclareMathOperator{\End}{End}
\DeclareMathOperator{\Tor}{Tor}
\DeclareMathOperator{\Ker}{Ker}
\DeclareMathOperator{\CoKer}{CoKer}
\DeclareMathOperator{\Spec}{Spec}
\DeclareMathOperator{\Proj}{Proj}
\DeclareMathOperator{\Lie}{Lie}
\DeclareMathOperator{\Map}{Map}
\DeclareMathOperator{\Endo}{End}
\DeclareMathOperator{\tr}{tr}
\DeclareMathOperator{\ad}{ad}
\renewcommand{\Im}{\mathop{\mathrm{Im}}}
%Their calligraphic versions; use these for the sheaf constructions
\DeclareMathOperator{\HHom}{\mathcal{H} \textit{om}}
\DeclareMathOperator{\EExt}{\mathcal{E} \textit{xt}}
\DeclareMathOperator{\EEnd}{\mathcal{E} \textit{nd}}
\DeclareMathOperator{\TTor}{\mathcal{T} \textit{or}}
\DeclareMathOperator{\KKer}{\mathcal{K}\textit{er}}
\DeclareMathOperator{\CCoKer}{\mathcal{C} \textit{o}\mathcal{K} \textit{er}}
\newcommand{\IIm}{\mathop{\mathcal{I} \textit{m}}}
\newcommand{\ccH}{\mathscr{H}} %The very curly H

%This makes alternating tensors look right in displayed equations
\newcommand{\Alt}{\bigwedge\nolimits}

\newcommand{\mf}[1]{\mathfrak{#1}}

%Blackboard bold letters.
\renewcommand{\AA}{\mathbb{A}}
\newcommand{\BB}{\mathbb{B}}
\newcommand{\CC}{\mathbb{C}}
\newcommand{\DD}{\mathbb{D}}
\newcommand{\EE}{\mathbb{E}}
\newcommand{\FF}{\mathbb{F}}
\newcommand{\GG}{\mathbb{G}}
\newcommand{\HH}{\mathbb{H}}
\newcommand{\II}{\mathbb{I}}
\newcommand{\JJ}{\mathbb{J}}
\newcommand{\KK}{\mathbb{K}}
\newcommand{\LL}{\mathbb{L}}
\newcommand{\MM}{\mathbb{M}}
\newcommand{\NN}{\mathbb{N}}
\newcommand{\OO}{\mathbb{O}}
\newcommand{\PP}{\mathbb{P}}
\newcommand{\QQ}{\mathbb{Q}}
\newcommand{\RR}{\mathbb{R}}
\renewcommand{\SS}{\mathbb{S}}
\newcommand{\TT}{\mathbb{T}}
\newcommand{\UU}{\mathbb{U}}
\newcommand{\VV}{\mathbb{V}}
\newcommand{\WW}{\mathbb{W}}
\newcommand{\XX}{\mathbb{X}}
\newcommand{\YY}{\mathbb{Y}}
\newcommand{\ZZ}{\mathbb{Z}}

%Calligraphic letters

\newcommand{\cA}{\mathcal{A}}
\newcommand{\cB}{\mathcal{B}}
\newcommand{\cC}{\mathcal{C}}
\newcommand{\cD}{\mathcal{D}}
\newcommand{\cE}{\mathcal{E}}
\newcommand{\cF}{\mathcal{F}}
\newcommand{\cG}{\mathcal{G}}
\newcommand{\cH}{\mathcal{H}}
\newcommand{\cI}{\mathcal{I}}
\newcommand{\cJ}{\mathcal{J}}
\newcommand{\cK}{\mathcal{K}}
\newcommand{\cL}{\mathcal{L}}
\newcommand{\cM}{\mathcal{M}}
\newcommand{\cN}{\mathcal{N}}
\newcommand{\cO}{\mathcal{O}}
\newcommand{\cP}{\mathcal{P}}
\newcommand{\cQ}{\mathcal{Q}}
\newcommand{\cR}{\mathcal{R}}
\newcommand{\cS}{\mathcal{S}}
\newcommand{\cT}{\mathcal{T}}
\newcommand{\cU}{\mathcal{U}}
\newcommand{\cV}{\mathcal{V}}
\newcommand{\cW}{\mathcal{W}}
\newcommand{\cX}{\mathcal{X}}
\newcommand{\cY}{\mathcal{Y}}
\newcommand{\cZ}{\mathcal{Z}}
\newcommand{\fg}{\mathfrak{g}}
\newcommand{\fh}{\mathfrak{h}}
\newcommand{\fn}{\mathfrak{n}}
\newcommand{\fb}{\mathfrak{b}}
\newcommand{\GL}{\text{GL}}

\newtheorem{thm}{Theorem}
\newtheorem{lem}[thm]{Lemma}
\newtheorem{prop}[thm]{Proposition}
\newtheorem{cor}[thm]{Corollary}

\theoremstyle{definition}
\newtheorem{defn}[thm]{Definition}
\newtheorem{expl}[thm]{Example}
\newtheorem{expls}[thm]{Examples}
\newtheorem{rmk}[thm]{Remark}
\newtheorem{example}[thm]{Example}

\title{MAT5105 study notes}
\author{Robert Lech}
\date{\today}

\begin{document}

\maketitle

\section{Planar Graphs}

% Jordan-Curve Theorem
\begin{thm} (Jordan-Curve Theorem)

A Jordan curve $J$ in the plane separates the rest of the plane into two disjoint open sets $int(J)$ and $ext(J)$ (with their respective closures $Int(J)$ and $Ext(J)$ intersecting in $J$). Any arc with origin in $int(J)$ and terminus in $ext(J)$ must intersect $J$ in at least one point.
\end{thm}


% Non-Planar
\begin{thm} (Non-Planar)

$K_5$ and $K_{3,3}$ are non-planar.
\end{thm}


% Proper Face Conditions
\begin{thm} (Proper Face Conditions)

If $G$ is non-separable, and $G\ncong K_1, K_2$ $\Rightarrow$ the boundary of each face is a cycle.
\end{thm}


% Swap Inner Face with Outer Face
\begin{prop} (Swap Inner Face with Outer Face)

Let $G$ be a planar graph with a planar embedding $\widetilde{G}$. Take any face $f\in F(\widetilde{G})$. Then $G$ has a planar embedding such that $\partial(f)$ is the boundary of the outer face.
\end{prop}


% Subdivision Planarity
\begin{prop} (Subdivision Planarity)

$G$ is a planar graph $\Longleftrightarrow$ every subdivision of $G$ is planar.
\end{prop}


% $K_5$ and $K_{3,3}$ Subdivisions are not Planar
\begin{cor} ($K_5$ and $K_{3,3}$ Subdivisions are not Planar)

Every subdivision of $K_5$ and $K_{3,3}$ is not planar.
\end{cor}


% [KURATOWSKI] NASC for Planarity
\begin{thm} (KURATOWSKI) (NASC for Planarity)

$G$ is a planar graph $\Longleftrightarrow$ $G$ contains no subdivision of $K_5$ or $K_{3,3}$.
\end{thm}


% NASC for Embeddability
\begin{thm} (NASC for Embeddability)

$G$ is a embeddable on a plane $\Longleftrightarrow$ $G$ is a embeddable on a sphere
\end{thm}


% Dual Facts
\begin{lem} (Dual Facts)

Let $G$ be a plane graph, $G^*$ its dual. Then:
\begin{itemize}
  \item $|V(G^*)|=|F(G)|$
  \item $|E(G^*)|=|E(G)|$
  \item $d_G(f)=d_{G^*}(f^*), \forall f\in F(G)$
\end{itemize}

\end{lem}


%Faces-Edge Relation
\begin{cor} (Faces-Edge Relation)

$\sum_{f\in F(G)} d_G(f)=2|E(G)|$
\end{cor}


%Euler's Formula
\begin{thm} (Euler's Formula)

$G$ connected graph on a surface with characteristic $\sigma$ $\implies$ $|V|-|E|+|F|=\sigma$.
\end{thm}


%Euler's Formula Corollary
\begin{cor} (Euler's Formula Corollary)

Let $G$ be a connected and planar. Then
\begin{itemize}
  \item All planar embeddings of $G$ have the same number of faces.
  \item If $G$ is simple and $n\geq 3$, then $|E| \leq 3|V|-6$.
  \item If $G$ is simple and $n\geq 3$, then $\delta(G) \leq 5$.
\end{itemize}
\end{cor}


%Non-Planar
\begin{cor} (Non-Planar)

$K_5$ and $K_{3,3}$ are non-planar.
\end{cor}


\section{Stable Sets, Cliques, and Colourings}

%Clique/Stable Set Connection
\begin{lem} (Clique/Stable Set Connection)

$S$ is a stable set of $G$ $\Longleftrightarrow$ $S$ is a clique of $\overline{G}$.
\end{lem}


%Lower Bound on $\chi$ with $\alpha$
\begin{lem} (Lower Bound on $\chi$ with $\alpha$)

$\chi \geq n/\alpha$
\end{lem}


%Upper on Chromatic Number for Subgraph $
\begin{lem} (Bound on Chromatic Number for Subgraph)

If $H \subseteq G$, then $\chi(H) \leq \chi(G)$
\end{lem}


%Lower Bound on $\chi$ with $\omega$
\begin{cor} (Lower Bound on $\chi$ with $\omega$)

$\chi \geq \omega$
\end{cor}


%Upper Bound on $\chi$ with $\Delta$
\begin{cor} (Upper Bound on $\chi$ with $\Delta$)

$\chi \leq \Delta+1$
\end{cor}


%Critical Implies Connected
\begin{lem} (Critical Implies Connected)

If $G$ is critical, then $G$ is connected.
\end{lem}


%Critical Implies Bound on $\delta$
\begin{thm} (Critical Implies Bound on $\delta$)

If $G$ is $k-$critical, then $\delta(G)\geq k-1$
\end{thm}


%Chromatic and Degree of Vertices
\begin{cor} (Chromatic and Degree of Vertices)

If $G$ is $k-$chromatic, then $G$ has $k$ vertices $(v_1,\ldots, v_k)$ such that $d(v_i)\geq k-1$
\end{cor}


%Upper Bound on $\chi$ with $\Delta$
\begin{cor} (Upper Bound on $\chi$ with $\Delta$)

$\chi \leq \Delta+1$
\end{cor}


%NC for $S-$component
\begin{thm} (NC for $S-$component)

If $G$ is $k-$critical and $S$ is a vx cut of $G$, then $S$ is not a clique.
\end{thm}


%NC for $\{u\}-$component
\begin{cor} (NC for $\{u\}-$component)

If $G$ is $k-$critical and $S=\{u\}$ is a vx cut of $G$, then $v$ is not a separating vertex.
\end{cor}


%NC for $\{u,v\}-$component
\begin{cor} (NC for $\{u,v\}-$component)

If $G$ is $k-$critical and $S=\{u,v\}$ is a vx cut of $G$, then $u\nsim v$.
\end{cor}


%[DIRAC] Type-1, Type-2 Subgraphs for $\{u,v\}-$components
\begin{thm} (DIRAC) (Type-1, Type-2 Subgraphs for $\{u,v\}-$components)

If $G$ is $k-$critical and $S=\{u,v\}$ is a vx cut of $G$, then:
\begin{itemize}
  \item $G=G_1+G_2$, where $G_1$ is type-1 and $G_2$ is type-2.
  \item $G+uv$ and $G_2/\{u,v\}$ are $k-$critical.
\end{itemize}
\end{thm}


%Critical Implies Bound on Degrees
\begin{cor} (Critical Implies Bound on Degrees)

Let $G$ is $k-$critical and $S=\{u,v\}$ is a vx cut of $G$, then $d_G(u)+d_G(v)\geq 3k-5$.
\end{cor}


%1,2,3-Critical Graphs
\begin{lem} ($1,2,3-$Critical Graphs)

Let $G$ is $k-$critical. Then:
\begin{itemize}
  \item $k=1\implies G \cong K_1$
  \item $k=2\implies G \cong K_2$
  \item $k=3\implies G \cong C_{2k+1}$
\end{itemize}
\end{lem}


%[Brooks 1941] Better Bound on $\chi$
\begin{thm} (BROOKS 1941) (Better Bound on $\chi$)

Let $G$ be a connected, simple graph where $G\ncong K_n, G\ncong C_{2l+1}$ then $\chi(G)\leq \Delta(G)$.
\end{thm}


\section{Matchings and Coverings}

% [BERGE] Augmenting Paths and Maximal Matching
\begin{thm} (BERGE) (Augmenting Paths and Maximal Matching)

Let $M$ be a matching of $G$. Then $M$ is maximum if and only if $G$ contains no $M$-augmenting path.
\end{thm}


% [TUTTE] Perfect Matchings and \# Odd Components with Vertex Cu
\begin{thm} (TUTTE) (Perfect Matchings and \# Odd Components with Vertex Cut)

Let $G$ be a connected, simple graph. $G$ has a perfect matching if and only if $o(G-S) \leq |S|$ for all $S \subseteq V$.
\end{thm}


% [PETERSEN] Perfect Matchings and 3-Regular Graphs
\begin{thm} (PETERSEN) (Perfect Matchings and 3-Regular Graphs)

If $G$ is a $3$-regular graph without cut edges, then $G$ has a perfect matching.
\end{thm}


% [HALL] NASC for Saturated Matching in Bipartite Graphs
\begin{thm} (HALL) (NASC for Saturated Matching in Bipartite Graphs)

Let $G$ be a bipartite graph with bipartition $(X,Y)$. Then $G$ contains a matching that saturates
every vertex in $X$ if and only if $|N(S)| \geq |S|$ for all $S \subseteq X$.
\end{thm}


% [MARRIAGE] Perfect Matching and Regular Bipartite Graphs
\begin{thm} (HALL/MARRIAGE THEOREM) (Perfect Matching and Regular Bipartite Graphs)

If $G$ is a $k$-regular bipartite graph with $k > 0$, then $G$ has a perfect matching.
\end{thm}


% Regular Graphs and 1-Factorizations
\begin{thm} (Regular Graphs and 1-Factorizations)

If $G$ is a $k$-regular graph with $k > 0$, then $G$ has a 1-factorization.
\end{thm}


% [K\"{O}NIG-EGERV\'{A}RY] Maximum Matching and Minimum Covering in Bipartite Graphs
\begin{thm} (K\"{O}NIG-EGERV\'{A}RY) (Maximum Matching and Minimum Covering in Bipartite Graphs)

If $G$ is a bipartite graph with a maximum matching $M^*$ and a minimum covering $\widetilde{K}$, then $|M^*|=|\widetilde{K}|$
\end{thm}


\section{Hamilton Graphs}


% Hamilton Cycle and \# Components with Vertex Cut
\begin{thm} (Hamilton Cycle and \# Components with Vertex Cut)

Let $G$ be a Hamilton graph. Then $\forall S\subset V$, $S\neq \emptyset$, $c(G-S)\leq |S|$.
\end{thm}

% [BONDY-CHVATAL] Conditions for when $G$ Hamiltonian $\Longleftrightarrow$ $G+uv$ Hamiltonian
\begin{lem} (BONDY-CHVATAL) (Conditions for when $G$ Hamiltonian $\Longleftrightarrow$ $G+uv$ Hamiltonian)

Let $G$ be a simple graph of order $n$. Let $u$ and $v$ be non-adjacent vertices with $d(u) + d(v) \geq n$. Then $G$ is Hamiltonian if and only if $G+uv$ is Hamiltonian.
\end{lem}


% [DIRAC] $\delta\geq \frac{n}{2}$ implies Hamiltonian
\begin{thm} (DIRAC 2) ($\delta\geq \frac{n}{2}\implies$ Hamiltonian)

Let $G$ be a simple graph of order $n \geq 3$ and minimum degree $\delta \geq n/2$. Then $G$ is Hamiltonian.
\end{thm}

% $cl(G)$ Closure Hamiltonian $\Longleftrightarrow$ $G$ Hamiltonian
\begin{cor} ($cl(G)$ Hamiltonian $\Longleftrightarrow$ $G$ Hamiltonian)

Let $G$ be a simple graph. Then $G$ is Hamiltonian if and only if the closure is Hamiltonian.
\end{cor}


% $cl(G)$ Complete $\implies$ $G$ Hamiltonian
\begin{cor} ($cl(G)$ Complete $\implies$ $G$ Hamiltonian)

Let $G$ be a simple graph with $n\geq 3$ and $cl(G)\cong K_n$. Then $G$ is Hamiltonian.
\end{cor}


% [CHVATAL]  Stupid Conditions $\implies$ Hamiltonian
\begin{thm} (CHVATAL) (Stupid Conditions $\implies$ Hamiltonian)

Let $G$ be a simple graph of order $n \geq 3$ with degree sequence $(d_1,d_2,\ldots,d_n)$, where
$d_1 \leq d_2 \leq \cdots \leq d_n$. Suppose that there is no value of $k$ less than $n/2$ for which
$d_k \leq k$ and $d_{n-k} \leq n-k$. Then $G$ is Hamiltonian.
\end{thm}

\section{Edge-Colourings}

% $\chi'$ and $\Delta$
\begin{lem} ($\chi'$ and $\Delta$)

Let $G$ be a graph. Then $\chi' \geq \Delta$.
\end{lem}

% [VIZING] Simple Graph, $\chi'$ and $\Delta$
\begin{thm} (VIZING) (Simple Graph, $\chi'$ and $\Delta$)

Let $G$ be a simple graph. Then $\chi' = \Delta$ (Class 1) or $\chi' = \Delta+1$ (Class 2).
\end{thm}


% Multiple Edge Graph, $\chi'$ and $\Delta$
\begin{lem} (Multiple Edge Graph, $\chi'$ and $\Delta$)

Let $G$ be a graph. Suppose $\mu=\#$ of edges between $u,v$, $\forall u,v\in V(G)$. Then $\chi' \leq \Delta+\mu$
\end{lem}


% Bipartite Graph, $\chi'$ and $\Delta$
\begin{thm} (Bipartite Graph, $\chi'$ and $\Delta$)

Let $G$ be a graph. If $G$ is bipartite, then $\chi' = \Delta$.
\end{thm}

\end{document}
